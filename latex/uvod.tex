Genetsko programiranje predstavlja optimizacijsku tehniku iz skupine evolucijskih algoritama. Ono omogućava rješavanje teških optimizacijskih problema za koje, ili ne postoji egzaktna matematička metoda rješavanja, ili su nerješivi u nekom konačnom vremenu. 

Temeljeći se na imitaciji prirodnog evolucijskog procesa, ovaj algoritam djeluje uporabom genetskih operatora križanja, mutacije i selekcije. Genetski operatori u svakoj iteraciji modificiraju trenutnu populaciju potencijalnih rješenja - jedinki. Pri tome, operatori mutacije i križanja služe za pretraživanje samog prostora rješenja, dok selekcija služi kako bi bolje jedinke imale veću vjerojatnost preživljavanja nad onim lošijim. Ovime, genetsko programiranje obuhvaća veći dio prostora pretrage nego srodni algoritmi, koji unutar prostora svih mogućih rješenja pretražuju samo ona koja trenutno susjedna. Još jedna velika prednost genetskog programiranja nad drugim tehnikama optimizacije je ta da je ono neovisno o domeni, odnosno, može se koristiti za rješavanje širokog skupa različitih problema.

Unatoč navedenim prednostima genetskog programiranja, kao i genetskih algoritama općenito, danas još uvijek nije jednoznačan postupak odabira specifičnih podvrsta genetskih operatora, kao ni samog prikaza potencijalnih rješenja. Postupak postavljanja algoritma oslanja se na iskustvo samog postavljača te se temelji na eksperimentiranju. Eksperimentiranjem, potrebno je utvrditi koji su genetski operatori, zajedno s njihovim specifičnim parametrima, dobri za rješavanje problema.

U ovom radu, dan je pregled i opis postojećih genetskih operatora s naglaskom na operatore križanja. Ispitana je učinkovitost algoritma genetskog programiranja s obzirom na odabir operatora križanja za tri najčešće vrste problema koji se rješavaju genetskim programiranjem - simboličku regresiju, pronalazak logičkih funkcija i programe.