Česta pojava u genetskom programiranju je prekomjeran rast jedinki kako teku generacije (\textit{eng. bloat}). Postoji više načina rješavanja ili sprječavanja ovog problema. Jedan od načina kako riješiti ovaj problem nad već generiranim jedinkama je podrezivanje (\textit{eng. pruning}), odnosno odstranjivanje onih dijelova stabla koji su izvan dopuštene dubine. Drugi način je da se rezultat operatora križanja odbacuje sve dok je stablo producirano takvim križanjem veće dubine nego što je dopušteno. Moguće je i kažnjavati predugačka rješenja umanjivanjem njihove dobrote. Ovaj operator taj problem rješava u samom postupku križanja. Naime, kontrolirajući veličinu podstabala koja ulaze u križanje, direktno kontrolira veličinu novogeneriranog potomka.

Nakon što je algoritam dosegnuo stagnaciju u napredovanju, uobičajeni operatori križanja pokušat će proširiti prostor pretrage. No, radi pojave prekomjernog rasta, ova potraga više će se protezati u prostoru duljine samog rješenja nego u prostoru različitih rješenja. Kod ovakvog pristupa, postoji veća vjerojatnost za preživljavanjem dijelova rješenja koji nisu štetni, ali nisu ni korisni - podržavajući tako daljnji rast jedinki. Kod križanja pravednog s obzirom na veličinu, to nije slučaj - nakon dosegnutog platoa, ono će istraživati različita rješenja, čisto iz razloga što je jedinka dobivena ovakvim križanjem otprilike jednake veličine kao i svoji roditelji.

U \cite{crxSizeFair} predstavljeno je križanje pravedno s obzirom na veličinu. Odabir točke prekida u prvom roditelju koji sudjeluje u križanju obavlja se nasumično. Pri tome, pristranost odabiru nezavršnog čvora stabla nad odabirom lista stabla iznosi 0.9. Ono po čemu se ovo križanje razlikuje od jednostavnog križanja je odabir točke prekida u drugom roditelju. Kako bi se pronašla druga točka prekida, izračunava se veličina stabla koje će biti obrisano u kopiji prvog roditelja. Nakon toga, u drugom stablu pronalaze se podstabla veličine ne veće od $1+2 \cdot |veličina\_obrisanog\_podstabla|$. Ovime se osigurava da novostvorena jedinka neće biti veća od $veličina\_prvog\_roditelja + |veličina\_obrisanog\_podstabla| + 1$.



\subsection{Dosadašnji rezultati}
U \cite{crxSizeFair}  provedeno je istraživanje nad četiri različita problema; po dva za simboličku regresiju (polinomi petog i šestog stupnja) i po dva za logičke funkcije (6 i 11 - multipleksor). Pokazano je kako je ovaj operator križanja kroz generacije proizvodio značajno, pa čak i do deset puta manje jedinke nego jednostavni operator križanja. Uočeno je da je u slučaju kada se početna populacija sastoji od jedinki koje su puno manje nego što je potrebno, jednostavno križanje i dalje superiornije pri pronalasku rješenja. Ovo se može pokazati kao značajan nedostatak ukoliko se na početku algoritma ne poznaje okvirna veličina dobrog rješenja. Dokazano je da ovaj operator značajno smanjuje prekomjeran rast jedinki. Za usporedbu, prosječan rast jedinki tijekom generacija za jednostavno križanje je linearno, dok je prosječan rast jedinki za ovo križanje subkvadratno.