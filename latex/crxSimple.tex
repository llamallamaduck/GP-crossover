Jednostavno križanje (\textit{eng. subtree crossover}) predstavljeno je u \cite{crxSimple}. Ovo križanje ujedno je i najjednostavnije i najkorištenije križanje u genetskom programiranju.

Nova jedinka nastaje spajanjem dvaju nasumično odabranih podstabala roditelja. U svakom roditelju odabire se po jedna točka prekida koja označava mjesta na kojima će se dogoditi križanje.

Često, odabir točaka prekida ne odvija se uniformno. Naime, tipični čvorovi stabala imaju prosječni faktor grananja \footnote{broj djece pojedinog čvora} (\textit{eng. branching factor}) jednak barem 2, što uzrokuje da je većina čvorova nekog stabla u stvari list, odnosno, završni znak. Posljedično, ukoliko bi ovo križanje biralo točke prekida na uniforman način, došlo bi do vrlo male količine razmijenjenog genetskog materijala: nerijetko bi se dogodilo da križanjem nastane nova jedinka koja je gotovo identična jednom od roditelja - od tog roditelja, razlikovala bi se u samo jednom listu koji pripada onom drugom roditelju. Kako bi se razriješio ovaj problem, Koza \cite{koza} je predložio kasnije često korišten pristup - prilikom odabira točke prekida, postavlja se vjerojatnost odabira nezavršnog čvora na 0.9, a vjerojatnost odabira lista stabla kao točke prekida na 0.1.

Ovakvim križanjem, dobivaju se dvije različite, nove jedinke. Iako neke implementacije ovog operatora uzimaju oba dva djeteta u novu generaciju, češće je slučaj da se uzima samo jedno dijete. Na slici \ref{crxSimple} prikazan je primjer jednostavnog križanja.

\begin{figure}[H]
 	\centering


\begin{tikzpicture}
	[sibling distance=25mm, level distance=15mm,
	every node/.style={fill=blue!20,circle,draw,drop shadow, minimum height=1cm}]
	
	
	\node  {\textbf{+}}
    		child {node {a}}
    		child {node [fill=green!20]  {\textbf{$sin$}}
        		child {node  {x}}
      		};
	};

\begin{scope}[xshift=7cm, yshift=0cm]
	\node {\textbf{+}}
    		child {node {$cos$}
        		child {node {y}}
      		}
     		child {node [fill=yellow!20]  {\textbf{$\cdot$}}
        		child {node {2}}
        		child {node {y}}
      		};
	};
\end{scope}

\begin{scope}[xshift=0cm, yshift=5cm]

	\node {\textbf{+}}
    		child {node {a}}
    		child {node [fill=yellow!20]  {\textbf{$\cdot$}}
        		child {node {2}}
        		child {node {y}}
      		};
	};
\end{scope}

\begin{scope}[xshift=7cm, yshift=5cm]
	\node {\textbf{+}}
    		child {node {$cos$}
        		child {node {y}}
      		}
    		child {node [fill=green!20] {\textbf{$sin$}}
			child {node {x}}};
	};
\end{scope}

\end{tikzpicture}


	\caption{Primjer jednostavnog križanja stabala}
	\label{crxSimple}
\end{figure}

Na prvom roditelju (gornji lijevi kut) odabrana točka prekida označena je žutom bojom, a na drugom roditelju zelenom bojom. Na donjem dijelu slike vidljivi su rezultati križanja - dva moguća potomka nastala križanjem ova dva roditelja. Lijevo dijete nastalo je spajanjem podstabla prvom roditelja koje se nalazilo iznad točke prekida i podstabla drugog roditelja koje se nalazilo ispod njegove točke prekida. Desno dijete nastalo je suprotnom kombinacijom - spojeno je podstablo ispod točke prekida prvog roditelja i podstablo iznad točke prekida drugog roditelja.

Budući da je ovaj operator križanja najstariji operator, u daljnjem tekstu uglavnom će služiti kao osnovna usporedba učinkovitosti. Za očekivati je da će svaki drugi operator predstavljen u idućim poglavljima donijeti poboljšanja u odnosu na ovaj operator.