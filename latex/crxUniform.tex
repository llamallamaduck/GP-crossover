Uniformno križanje predstavljeno je u \cite{crxUniform}. Inspirirano je uniformnim križanjem nestablastih struktura u genetskom algoritmu. U genetskom algoritmu, ovo križanje (budući da su sve jedinke jednakih dužina) prolazi kroz oba roditelja te na svakoj poziciji nasumično odabire gen iz prvog ili drugog roditelja te ga ugrađuje u dijete. Kada bi u genetskom programiranju sve jedinke bile potpuno jednakog oblika, ovo križanje funkcioniralo bi potpuno jednako. No, budući da je to vrlo rijedak slučaj, potrebno je, kao i u križanju s jednom točkom prekida, pronaći zajedničko područje dva roditelja. Nakon što je to područje pronađeno, s lakoćom se smiju izmjenjivati čvorovi roditeljskih stabala - no pritom je potrebno paziti na broj djece koji taj čvor posjeduje. Na slici \ref{crxUni} prikazan je primjer jednog mogućeg rezultata uniformnog križanja.

\begin{figure}[H]
 	\centering


\begin{tikzpicture}
	[sibling distance=25mm, level distance=15mm,
	every node/.style={fill=blue!20,circle,draw,drop shadow, minimum height=1cm}]

\begin{scope}[xshift=0cm]

	\node  [fill=yellow!20]  {\textbf{+}}
    		child {node [fill=yellow!20]  {$cos$}
    			child {node  [fill=yellow!20] {-}
    				child {node{x}}
    				child {node{y}}
    			}
    		}
    		child {node [fill=yellow!20]  {\textbf{$\cdot$}}
        		child {node [fill=yellow!20]  {2}}
        		child {node [fill=yellow!20]  {y}}
      		};
	};
\end{scope}

\begin{scope}[xshift=7cm]
	\node  [fill=yellow!20]  {\textbf{+}}
    		child {node  [fill=yellow!20] {$sin$}
        		child {node  [fill=yellow!20] {y}}
      		}
    		child {node [fill=yellow!20] {\textbf{$/$}}
			child {node [fill=yellow!20]  {x}}
			child {node [fill=yellow!20]  {y}}	
		};
	};
\end{scope}

\end{tikzpicture}


	\caption{Zajedničko područje dva roditelja}
	\label{crxOnePointCommon}
\end{figure}

\begin{figure}[H]
 	\centering


\begin{tikzpicture}
	[sibling distance=25mm, level distance=15mm,
	every node/.style={fill=blue!20,circle,draw,drop shadow, minimum height=1cm}]

\begin{scope}[xshift=0cm]

	\node {\textbf{+}}
    		child {node [fill=red!20]  {$sin$}
    				child {node{-}
    					child {node {x}}
    					child {node {y}}
    			}
    		}
    		child {node  [fill=red!20] {\textbf{$/$}}
        		child {node [fill=red!20]  {x}}
        		child {node {y}}
      		};
	};
\end{scope}

\begin{scope}[xshift=7cm]
	\node   {\textbf{+}}
    		child {node  [fill=red!20] {$sin$}
        		child {node  [fill=red!20] {y}}
      		}
    		child {node  [fill=red!20]  {\textbf{$\cdot$}}
			child {node  {x}}
			child {node  {y}}	
		};
	};
\end{scope}

\end{tikzpicture}


	\caption{Mogući rezultati križanja roditelja sa slike \ref{crxOnePointCommon}}
	\label{crxUni}
\end{figure}

Velika prednost ovog križanja je velika stopa razmjene genetskog materijala. Budući da se čvorovi koji će završiti u djetetu biraju uniformno, s omjerima vjerojatnosti 1:1, velika je vjerojatnost da će dijete sadržavati 50\% genetskog materijala prvog roditelja i 50\% genetskog materijala drugog roditelja. Ovo je vrlo pogodno za brzo napredovanje algoritma.

\subsection{Dosadašnji rezultati}
Autor ovog križanja proveo je eksperimente nad paritetnim problemom 4 varijable. Cilj ovog problema je pronaći logičku funkciju koja je istinita za parni broj istinitih varijabli. Tablica istinitosti za ovaj problem prikazana je na tablici ispod (\ref{truthTable}).

\begin{table}[H]
\centering

\begin{tabular} {a b c d | p}
A & B & C & D & P \\
\hline
0 & 0 & 0 & 0 & 1 \\
0 & 0 & 0 & 1 & 0 \\
0 & 0 & 1 & 0 & 0 \\
0 & 0 & 1 & 1 & 1 \\
0 & 1 & 0 & 0 & 0 \\
0 & 1 & 0 & 1 & 1 \\
0 & 1 & 1 & 0 & 1 \\
0 & 1 & 1 & 1 & 0 \\
1 & 0 & 0 & 0 & 0 \\
1 & 0 & 0 & 1 & 1 \\
1 & 0 & 1 & 0 & 1 \\
1 & 0 & 1 & 1 & 0 \\
1 & 1 & 0 & 0 & 1 \\
1 & 1 & 0 & 1 & 0 \\
1 & 1 & 1 & 0 & 0 \\
1 & 1 & 1 & 1 & 1 \\
\hline
\end{tabular}

	\caption{Tablica istinitosti za paritetni problem 4 varijable}
	\label{truthTable}
\end{table}

Eksperimentima je dokazano kako za razliku od jednostavnog križanja i križanja s jednom točkom prekida, u kojima je prosječna stopa razmjene genetskog materijala oko 5\%, ovo križanje uistinu ima konstantnu stopu razmjene genetskog materijala od oko 50\%.

Pokazano je kako jednostavno križanje pretragu prostora rješenja odvija lokalno, nasljeđujući većinu genetskog materijala od jednog roditelja. Jednostavno križanje nije idealno za brzo pretraživanje prostora - bolje je za fino ugođavanje prilično dobrog rješenja. Osim toga, jednostavno križanje vrlo lako zapne u nekom od lokalnih optimuma. 

Križanje s jednom točkom prekida bolje se nosi s veličinom prostora pretrage, no nakon nekog vremena počinje sagledavati sve manji dio prostora - lokalizirajući tako pretragu s povećanjem vjerojatnosti zapinjanja u lokalnom optimumu.

Uniformno križanje prelazi preko ovih problema. Za razliku od jednostavnog križanja, ono nije pristrano na križanje listova ili podstabala. Ovime omogućuje slobodniju i raznolikiju izmjenu genetskog materijala između roditelja, ubrzavajući tako konvergenciju rješenja u globalni optimum. Također, pokazano je kako nakon nekog vremena, sve jednike populacije budu otprilike jednake veličine, što kod upotrebe jednostavnog križanja i križanja s jednom točkom prekida nije slučaj. Osim navedenih svojstava, nije uočena prevelika razlika u samoj uspješnosti križanja da pronađe dobro rješenje problema - rad se pretežito fokusirao na razliku u količini izmijenjenog genetskog materijala.

