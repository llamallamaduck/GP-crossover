Križanje je daleko najvažniji operator evolucijskih algoritama. Bez njega, ne bi bila moguća rekombinacija korisnog genetskog materijala u neki novi, bolji materijal sljedeće generacije. Budući da je danas razvijen poveći broj operatora križanja, pitanje koje se postavlja prilikom formulacije problema kojega je potrebno riješiti genetskim programiranjem je sam odabir operatora križanja. 

Ovaj rad pozabavio se upravo tim pitanjem. Usporedivši međusobno operatore križanja za tri najčešće rješavana problema genetskim programiranjem - pronalaska logičkih funkcija, simboličke regresije i strojnog učenja, dan je širok pregled učinkovitosti svakog od operatora, kako na pojedini problem, tako i na skup svih problema ispitanih u sklopu ovog rada. 

Zanimljiv zaključak provedenog istraživanja je činjenica da se u većini slučajeva naj-učinkovitijim operatorom pokazao jednostavni operator križanja. Ovaj operator je prvi operator u genetskom programiranju, te je cilj mnogih bio pronaći njegovo poboljšanje. Tako su nastali ostali operatori koji su, kada su predstavljeni, specifičan problem rješavali bolje nego jednostavni operator križanja. Iako su u tim, specifičnim slučajevima donosili određena poboljšanja, ovaj rad pokazao je kako je sigurno dobra polazišna točka u rješavanju nekog problema genetskim programiranjem korištenje jednostavnog operator križanja.